%HW 2 answer .tex file

\documentclass{article}
\usepackage{times}
\usepackage{graphicx}
\usepackage{amsmath}

\begin{document}

\title{c++11 :: tips and short notes } 
\author{Pintu Lal M\\
        \site(http://cse.iitb.ac.in/~pin2lalmee/) \\}
} 
\date{\today}  
\maketitle

\begin{abstract}
This article is a short note type article about diffirect c++ starndard use and their functionallity .

\end{abstract}


\section{Class }
class Obj\\
\{\\
  public:  \\
  Obj(const Obj& obj); // copy constructor  \\
  Obj(Obj&& obj); move constructor  \\
  
  Obj& operator=(const Obj& obj) // copy assignment operator\\
  Obj& operator=(Obj&& obj)  // move assignment operator\\
\}\\

\subsection{difference bitween copy constructor and move constructor}
 copy constructor copies the values of a copy object to new object \\
 while move constructor give a reference to the internal object to new object so there is no copy overhead \\
 
 example: \\
 supporse object is a atring array :\\
 then \\ 
  copy constructor creates a new copy of array then assign it to new object while move constructor assign the same array to new object without creating any copy of object .\\
  
 so move constructor can reduce the object copy overhead thus performance improvement\\
 

\begin{figure}
  \includegraphics[width=\linewidth]{1.jpg}
  \caption{2-dtree.}
  \label{fig::2d_tree}
\end{figure}
Figure \ref{fig::2d_tree} .

2-d binary tree has log(n) height so conbined nodes has log(n)/2 height .\\

G(n) = 2+2*(2+2*G(n/16))\\
G(n) = 2 +2^2+2^2 *G(n/16)\\
G(n) = 2 + ........2^\[\log(n)/2\] * G(1)\\
G(n) = 2(1+........2^\[\log(n)/2-1\] ) \\
G(n) = 2*2^\[\log(n)/2\] \\
G(n) = 2*2^\[\log(\sqrt(n)) \]\\
G(n) = 2\sqrt(n) \\
G(n) = O(\sqrt(n))\\
 
For every report event , time O(1)\\
so if there are k reported leaf node G(n) = O(\sqrt(n)+k)\\


\end{document}  %End of document.
